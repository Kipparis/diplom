\Introduction

Выпускная квалификационная работа посвящена поиску оптимальных с точки зрения
масштабирования и эксплуатации компонентов системы, их коммуникации и дальнейшей
виртуализации с целью поиска алгоритмов по имени на интернет ресурсах без
использования официального API. Объектом исследования является поиск и настройка
оптимального архитектурного решения, которое подходит не только под современные
требования, но и хорошо укладывается в общую концепцию цели ВКР.

Технологии достаточно плотно проникли в образ жизни среднестатистического
человека. Быстро развиваясь, не каждый способен уследить весь прогресс, который
поддерживает нас ежедневно. Почти у каждого есть ЭВМ которая помещается в
карман, или даже надевается на руку. Это все значительно упрощает жизнь в том
плане, что человеку больше не нужно отправлять почтовых голубей и решать
проблемы с ними связанными. Или же, теперь не нужно ждать пока друг вернется
домой и наконец-то возьмет городской телефон. В текущее время, ты можешь быть
всегда на связи, принимать самые оперативные решения и максимально свободен в
доступе к информации.

Но есть одно но. Удобство технологий и простота их использования достается не
бесплатно. Каждый новый программный продукт является потенциальным "кирпичиком"
для следующих разработок. Данный "кирпичик" необходимо поддерживать: производить
обновления безопасности, внедрять новый актуальный функционал (например, если
программа работает с видеофайлами, то внедрять новые кодеки).

В 21 веке мы пользуемся такими программами, о которых в 20 веке и не подумали
бы. Все это, более чем напрямую затрагивает и самих программистов. Когда-то люди
использовали перфокарты, сейчас - специальные среды разработки с подсветкой
синтаксиса, автодополнением, линкером и copilot для разработки на
высокоуровневом языке программирования.

Одной из недостающих утилит для современного разработчика является возможность
быстрого поиска алгоритмов. Обычно эта задача решается "в лоб": пытаемся найти в
интернете аналогичную задачу; открываем книжку по алгоритмам и структурам данным
и пытаемся найти что-то аналогичное и т.п.

Целью данной работы является разработка программного обеспечения для быстрого
развертывания инфраструктуры по поиску алгоритмов на интернет ресурсах,
находимых в общем доступе. Такое ПО будет не только находить алгоритмы, но и
предоставлять возможность по управлению сохраняемыми данными.

