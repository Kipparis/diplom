\Introduction

В современном мире наблюдаются следующие тенденции:
\begin{itemize}
    \item Люди нетерпеливы и привыкли к легкому и быстрому доступу к информации
    \item Количество доступной информации неукротимо растет и человек не в состоянии
    справиться с ее потоком без использования поисковиков
    \item Существенная часть информации -- аудиофайлы
\end{itemize}

Конкретный пример: люди, посещающие различные музыкальные мероприятия, часто сталкиваются
с ситуацией, когда на сцене выступает музыкант, а название песни
или даже имя исполнителя неизвестно (например, на фестивале).
Конечно, можно спросить ближайшего человека, но в таких местах обычно очень шумно.
Кроме того нет гарантий, что у этого человека найдется ответ на вопрос.
Существует множество методов и сервисов для нахождения музыкальных произведений по отрывку,
однако у них есть ряд ограничений:
\begin{itemize}
    \item Сервисы вроде Shazam способны искать только те записи, которые уже есть в базе
    \item Некоторые сервисы умеют искать произведения по мелодии, но у них
            довольно низкая точность.
    \item Сервисы, которые ищут каверы или ремиксы (для защиты авторских прав) не приспособлены
            к нахождению зашумленных отрывков, поскольку предполагают, что кавер записывался в
            студийных условиях
\end{itemize}
В этой работе рассмотрен метод, лишенный всех вышеобозначенных недостатков.
Поскольку музыкальных произведений в мире очень много (например, в социальной сети VK 400 миллионов треков),
то очень важно уметь быстро и эффективно по памяти обрабатывать эти данные.
Кроме того этот метод применим не только к песням, так как аудиофайл представляет собой временной ряд.\\
Цели данной работы:
\begin{enumerate}[label=\arabic*.]
    \item Разработать библиотеку, которая предоставляла бы гибкий и удобный интерфейс
            для эффективной обработки и поиска аудиофайлов
    \item Реализовать клиент для идентификации концертных записей, используя разработанную
    библиотеку
\end{enumerate}
