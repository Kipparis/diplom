\chapter{Практическая часть}
\label{cha:ch_2}

\section{Проектирование архитектуры}
\subsection{Use-Case диаграмма}
\subsection{Архитектура приложения}
% Какая задача поставлена
На основе данной задачи выделяются следующие подзадачи, которые необходимо
выполнить, для успешной реализации инфраструктуры:
\begin{enumerate}[label=\arabic*.]
    \item Описать возможные процессы взаимодействия между компонентами системы.
        Удобно произвести описание с помощью Use-Case диаграммы;
    \item Смоделировать архитектуру приложения: какие компоненты будут
        взаимодействовать друг с другом;
    \item Далее равнозначные задачи, которые не обязательно выполнять
        последовательно: написать файлы запуска и написать кодовую базу для
        запуска каждой из компоненты системы.
\end{enumerate}

Далее рассмотрим каждый из пунктов подробнее, опишем способ реализации,
возникшие проблемы и пути решения.

\section{Разработка практических решений}
\section{Docker}
\subsection{Docker Compose}
\section{Scrapy}
\subsection{Scrapyd}
\section{Flask}
\subsection{Nginx}
\subsection{Uwsgi}
\subsection{Jinja}
\section{Kafka}
\subsection{Zookeeper}
\subsection{Kafka Connect}
\section{Elastic Search}
