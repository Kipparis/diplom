\begin{center}
РЕФЕРАТ
\end{center}

% \large Московский авиационный институт\\[5.5cm]

% \huge Реферат \\[0.6cm] % название работы, затем отступ 0,6см
% \large на тему:  <<Метод идентификации музыкальных
% произведений по аудио фрагментам концертных исполнений>>\\[3.7cm]


% \end{center}

% \begin{flushright}
% Выполнил: студент гр. М8О-406Б \\
% Давид Гринберг \\
% \end{flushright}


% \vfill

% \begin{center}
% \large Москва 2020
% \end{center}

% \thispagestyle{empty}
Выпускная квалификационная работа содержит \pageref{LastPage} страниц, \totfig{}
рисунков, \total{citnum}\ использованных источников.

\noindent КОНТЕЙНЕРИЗАЦИЯ, БАЗЫ ДАННЫХ, БРОКЕРЫ СООБЩЕНИЙ, ОБРАБОТКА СТРАНИЦ, ИНТЕРНЕТ
СЕРВЕР, API, СИСТЕМА ШАБЛОНОВ ДЛЯ ДОСТАВКИ HTML СТРАНИЦ, UWSGI, NGINX, ELASTIC
SEARCH, KAFKA, PYTHON

Выпускная квалификационная работа посвящена поиску оптимальных с точки зрения
масштабирования и эксплуатации компонентов системы, их коммуникации и дальнейшей
виртуализации с целью поиска алгоритмов по имени на интернет ресурсах без
использования официального API.

В теоретической части рассматриваются принципы по которым должна строится
система, описываются выбранные компоненты и их структура. Также производится
сравнение с другими аналогичными продуктами и обосновывается сделанный выбор.

В практической части рассматриваются достигнутая архитектура, методы
отслеживания работы. Кроме того демонстрируется работа системы в целом и каким
способом ее можно маштабировать.
